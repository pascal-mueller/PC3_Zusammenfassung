% !TEX options=--shell-escape
\title{Zusammenfassung - PC3}
\author{
        Pascal Müller [pamuelle@student.ethz.ch]
}
\date{\today}

\documentclass[12pt]{article}
\usepackage[utf8]{inputenc}
\usepackage[parfill]{parskip}
\usepackage{amsmath}
\usepackage{xcolor}
\usepackage{physics}
\usepackage[margin=0.5in]{geometry}
\usepackage{multicol}
\usepackage{fourier}
\usepackage{graphicx}
\graphicspath{ {./img/} }

\definecolor{bkg_light}{HTML}{d0ddf2}
\definecolor{bkg_dark}{HTML}{717ebf}

\newcommand{\mysection}[1]{
    \setlength\fboxsep{4pt} %% spacing around box contents
    \section*{\colorbox{bkg_dark}{\makebox[0.47\textwidth]{\color{white}#1}}}
}

\newcommand{\mysubsection}[1]{
    \setlength\fboxsep{4pt} %% spacing around box contents
    \subsection*{\colorbox{bkg_light}{\makebox[0.47\textwidth]{\color{black}#1}}}
}

%\usepackage[outputdir=/home/pascal/.config/sublime-text-3/Cache]{minted}

\begin{document}
\maketitle


\subsection*{STATUS}

Fertige Serien: 1, 2, 3, 4, 5, 6

\paragraph{TODO:}
\begin{itemize}
  \item Add klassische Newton gesetzte?
  \item S5A2 (Spin Zeug)
  \item S6A1 (Gram Schmidt)
  \item S7A1 - Molekülschwingung
\end{itemize}

\newpage

\begin{multicols*}{2}
\mysection{Definitionen}
\mysubsection{Symbole:}
\begin{itemize}
  \item $\ket{\Psi}$ Zustandsvektor \\ $\Rightarrow$ Beschreibt ein System
  \item $\hat{O}$ Quantenoperator \\ $\Rightarrow$ Physk. Messbare Grösse
  \item $\hat{O}\ket{o_{spec}} = o_{spec}\ket{o_{spec}}$ \\ $\Rightarrow$ Wert
    den wir bei Messung eines Experiments bekommen würden.
  \item $\ket{\Psi} = \sum_i a_i\ket{o_i}$ \\
    $a_i$: Wahrsch. Amplitude \\
    $|a_i|^2$ Wahrsch. $o_i$ zu messen
  \item $\bra{\Psi}$ \\ $\Rightarrow$ Konjugierter Zustandsvektor
\end{itemize}

\mysubsection{Fermion:} (symm. Eigenfunktionen) \\
Spin = $0, 1, 2, \dots$

\mysubsection{Boson:} (antisymm. Eigenfunktionen) \\
Spin = $1/2, 3/2, 5/2, \dots$

\mysubsection{Observable:} (Physik. messbare Grössen). \\
Beschrieben durch Operatoren (Matrizen sind Operatoren).

\mysubsection{Kommutator:} $[\hat{A},\hat{B}] = \hat{A}\hat{B} - \hat{B}\hat{A}$
\begin{itemize}
  \item \warning Wende Testfunktion an beim Berechnen!
  \item Falls $[\hat{A},\hat{B}] = 0 \ \ \Rightarrow \ \ $ gemeinsame Basis und somit
gleichzeitig messbar.
\end{itemize}

\mysubsection{Orbital:} TODO

\mysubsection{Quadratisch Integrierbar:} Eine Funktion $f(x)$ heiss Quad. Int. bar
falls $\int_\infty^\infty f^{*}(x) f(x) dx < \infty$

\mysubsection{Korrespondenzprinzip:} (Gleich für $x,y,z$) \\
Position: $\hat{x} = x$ \\
Impuls: $\hat{p}_x = -i\hbar\frac{\partial}{\partial x}$

\mysubsection{Normierung} $\bra{\Psi}\ket{\Psi} = 1$

\mysubsection{Matrixform} Der Operator $\hat{A}$ kann als Matrix dargestellt
werden bezüglich einer Basis $\varphi_i$.

\[
  A_{nm} = \int_{\text{Raum}}\varphi_n^{*}\hat{A}\varphi_m dx 
         = \bra{n}\hat{A}\ket{m}
\]

\mysubsection{Erwartungswert}: \\
Normiert:
\[
  \bar{A} = \expval{\hat{A}} = \int \Psi^{*}\hat{A}\Psi d\tau
\]

Nicht-Normiert
\[
  \expval{\hat{A}}
    = \frac{\int \Psi^{*}\hat{A}\Psi d\tau}{\int\Psi^{*}\Psi d\tau}
\]

\mysubsection{Zeitentwicklung von Erwartungswerten}
Die Zeitentwicklung von $\expval{\hat{A}}$ eines zeitunabhängigen Operators
$\hat{A}$ ist gegeben durch:

\[
  \frac{d\expval{\hat{A}}}{dt}
  = \frac{i}{\hbar}\expval{\comm{\textcolor{red}{\hat{H}}}{\hat{A}}}
\]

\warning 
$\expval{\hat{A}} = 0 \quad \Leftrightarrow \quad \comm{\hat{H}}{\hat{A}} = 0$

\mysubsection{Hermitesche Polynome}

TODO: Hinzufügen. Siehe Skript und S6 Tabelle 1.


\mysection{Theorem von Ehrenfest} (S5A1, §3.15) \\
Annahme: Teilchen der Masse $m$ bewegt sich in 1D Raum, dann

\[
  \hat{H} = \frac{\hat{p}_x^2}{2m} + V(x)
\]

wobei $V(x)$ beliebige Potentialfunktion. Dann:

\[
  \frac{d\expval{\hat{p}_x}}{dt} = - \expval{\frac{dV(x)}{dx}}
\]

\[
  \frac{d\expval{x}}{dt} = \frac{1}{m}\expval{\hat{p}_x}
\]

\warning Entspricht klassischen Ergebnisen! (1. Newton Gesetzt)

\paragraph{Bsp: Freies Teilchen} (S5A1.c) \\
$\Psi_{+k} = A \exp{ikx}$ und $V(x) = 0$ und somit 

\begin{align*}
  \frac{d\expval{\hat{p}_x}}{dt} &= - \expval{\frac{dV(x)}{dx}} = 0
  \rightarrow \expval{\hat{p}_x} \text{ist konstant}
  \\
  \frac{d\expval{x}}{dt} &= \frac{\hbar k}{m}
\end{align*}

\paragraph{Bsp: Teilchen im 1D Kasten} (S5A1.c) \\
$\varphi_m(x) = \sqrt{\frac{2}{L}} \sin{\big(\frac{n\pi x}{L}\big)}$ mit
$V(x) = 0$ für $0 < x < L$:

Wir wissen $\expval{\hat{p_x}} = 0$ (Serie 3) und somit

\begin{align*}
  \frac{d\expval{\hat{p}_x}}{dt} &= - \expval{\frac{dV(x)}{dx}} = 0
  \\
  \frac{d\expval{x}}{dt} &= \frac{1}{m}\expval{\hat{p}_x} =  0
\end{align*}

\paragraph{Bsp: Harmonischer Oszillator} (S5A1.c, S6A2, S7A1) \\
$\varphi_{\textcolor{red}{0}} = A\exp{-bx^2}$ und $V(x) = \frac{1}{2}kx^2$.

Wir wissen $\expval{\hat{x}} = 0$ und $\expval{\hat{p}_x} = 0$ (Serie 4) und 
somit

\begin{align*}
  \frac{d\expval{\hat{p}_x}}{dt} &= - \expval{\frac{dV(x)}{dx}}
    = k\expval{\hat{x}} = 0
  \\
  \frac{d\expval{x}}{dt} &= \frac{1}{m}\expval{\hat{p}_x} =  0
\end{align*}

\warning Oft benutzt als erste Näherung zur Beschreibung der
Schwingungsbewegung von Molekülen.


\mysection{Unbestimmtheitsrelationen}
\mysubsection{Ort-Impuls}
\[
  \Delta x \Delta p_x \geq \frac{\hbar}{2}
\]

mit

$\Delta x := \sqrt{\expval{\hat{x}^2} - \expval{\hat{x}}^2}$

$\Delta p_x := \sqrt{\expval{\hat{p_x}^2} - \expval{\hat{p_x}}^2}$

\mysection{Theoreme}

\mysection{Postulate}

\mysection{Akzeptable Wellenfunktion}

\mysection{Tunnel Effekt}


\mysection{Term Symbole}

\mysection{Variationsprinzip} (S4A3) (§3.13) \\
Der Erwartungswert von $\hat{H}$ bezüglich einer bel. Funktion $\Psi$ erfüllt

$\Psi$ normiert:
\[
  \bra{\Psi}\hat{H}\ket{\Psi} \geq E_1
\]

$\Psi$ nicht normiert:
\[
  \frac{\bra{\Psi}\hat{H}\ket{\Psi}}{\bra{\Psi}\ket{\Psi}} \geq E_1
\]

\warning Die LHS obiger Gleichungen ist somit eine obere Grenze für $E_1$. \\
\warning Beim "normalen" Variationsprinzip stopfen wir irgendeine Testfunktion
$\Psi$ in die LHS. Z.B. muss diese Testfunktion garkeinen Parameter haben.

\mysubsection{Ritzsche Variationsverfahren} (Testfunktion MIT Parameter) \\
Die Testfunktion $\Psi$ hat immer einen oder mehrere Parameter. Wir können dann
die LHS minimieren durch:

$\Psi$ normiert:
\[
  \frac{d}{da}\bra{\Psi}\hat{H}\ket{\Psi} \geq E_1
\]

$\Psi$ nicht normiert:
\[
  \frac{d}{da}\frac{\bra{\Psi}\hat{H}\ket{\Psi}}{\bra{\Psi}\ket{\Psi}} \geq E_1
\]

wobei $a$ unser Parameter ist. 

\mysection{Störungsrechnung}

\newpage

\mysection{Beispiele:}
\mysubsection{Teilchen im 1D Kasten} ($\hat{H}\varphi_n = E_n \varphi_n$) (S4A2.a)
\warning Beschreibt alle möglichen Zustände aufs mal! Fixieren von $n$
fixiert Zustand!

\begin{align*}
  \hat{H} &= \frac{-\hbar^2}{2m}\frac{d^2}{dx^2} + V(x)
  \\
  V(x) &= \begin{cases}0 &, 0 < x < L \\ \infty &, \text{else}\end{cases}
  \\
  \varphi_n(x) &=
    \begin{cases}
      \sqrt{\frac{2}{L}}\sin{\frac{n\pi x}{L}} & 0 < x < L \\
      0 & \text{else}
    \end{cases}
  \\
  E_n &= \frac{n^2 h^2}{8 m L^2}
  \\
  n &= \textcolor{red}{1},2,3,4, \dots 
\end{align*}

\includegraphics[width=0.5\textwidth]{1D\_kasten.jpg}

\warning Endliche Nullpunktsenergie: 
$E_{\textcolor{red}{1}} = \frac{h^2}{8mL^2} > 0$

\begin{itemize}
  \item $\expval{x} = \frac{L}{2}$
  \item $\expval{x^2} = \big(\frac{1}{3} - \frac{1}{2n^2\pi^2}\big)L^2$
  \item $\expval{p_x} = 0$
  \item $\expval{p_x^2} = \big(\frac{n\pi\hbar}{L}\big)^2$
\end{itemize}

\mysubsection{Harmonischer Oszillator} (S4A2.b, S6A2, §4.4) \\

\warning Molekülschwingung hat eigenen Teil in der Zsmfg!

\begin{align*}
  \hat{H} &= -\frac{\hbar^2}{2\textcolor{red}{m}}\frac{d^2}{dx^2} + 
  \underbrace{\frac{1}{2}\textcolor{green}{k}x^2}_{\text{V(x)}}
  \\
  \alpha &= \sqrt{\frac{mk}{\hbar^2}} = \frac{2\pi\textcolor{blue}{\nu} m}{\hbar} = 
            \frac{\omega m}{\hbar}
  \\
  \textcolor{blue}{\nu} &= \textcolor{blue}{\nu}_{\text{osz}} 
      = \frac{1}{2\pi}\sqrt{\frac{\textcolor{green}{k}}{m}}
  \\
  E_v &= h\textcolor{blue}{\nu}(v+\frac{1}{2}) \quad \text{für QZ} \quad 
  v = 0, 1, 2, \dots
  \\
  \Psi_v(x) &= (\alpha/\pi)^{1/4}(2^v v!)^{-1/2}H_v(\sqrt{\alpha} x)
              e^{-\alpha x^2 / 2}
\end{align*}


TODO: Warte auf Antwort per E-Mail bzgl. eventuelliger Fehler in Eig. func..

\warning Siehe Tabelle 4.1 (S. 4-13) für explizite $\Psi_v$. 

\paragraph{\underline{$\mathbf{v=0}$:}}

\begin{align*}
  \Psi_1(x) &= \big(\frac{\alpha}{\pi}\big)^{1/4} \exp{-\frac{\alpha x^2}{2}}
  \\
  E_0 &= \frac{1}{2} h \textcolor{blue}{\nu}
\end{align*}

\paragraph{\underline{$\mathbf{v=1}$:}}

\begin{align*}
  \Psi_1(x) &= \big(\frac{4\alpha^3}{\pi}\big)^{1/4} x 
  \exp{-\frac{\alpha x^2}{2}}
  \\
  E_1 &= \frac{3}{2}h \textcolor{blue}{\nu}
\end{align*}

\paragraph{\underline{$\mathbf{v=2}$:}}

\begin{align*}
  \Psi_2(x) &= \big(\frac{\alpha}{4\pi}\big)^{1/4} (1 - 2\alpha x^2) 
  \exp{-\frac{\alpha x^2}{2}}
  \\
  E_2 &= \frac{5}{2}h \textcolor{blue}{\nu}
\end{align*}

\includegraphics[width=0.5\textwidth]{harm\_osz.jpg}

Graphische Darstellung der Energieniveaus $E_v$ und der Wellenfunktion
$\Psi_v(x)$ (\textbf{schwarze Kurven}) respektive deren Betragsquadrate
$|\Psi_v(x)|^2$ (\textbf{graue Kurven}) eines eindimensionalen harmonischen
Oszillators. Zu beachte ist, dass auch im energetisch verbotenen Bereich
(\textbf{grau markiert}), in denen die totale Energie $E_v$ kleiner ist als die
potentielle Energie $V(x)$, eine von null verschiedene 
Aufenthaltswahrscheinlichkeit existiert.


\mysubsection{Schwingung 2-Atomiger Moleküle}

TODO: Machen. Siehe PVK Skript, normales Skript. Bis jetzt kam noch keine
direkte Aufgabe in den Serien ausser S7A1, jedoch nur halbpatzig.



\section{Operatoren}


\section{Kommutatoren}
\warning Für Eigenschaften von Kommutatoren, siehe Serie 3, Aufgabe 2.

\begin{align*}
  [\hat{x}, \hat{p}_x] &= -i\hbar
\end{align*}


\section{Trigonometrische Identitäten}
$\sin \alpha \sin \beta
  = \frac{1}{2}[\cos(\alpha - \beta) - \cos(\alpha + \beta)]$

$\sin \alpha \cos \beta
  = \frac{1}{2}[\sin(\alpha - \beta) - \sin(\alpha + \beta)]$

\end{multicols*}
\end{document}
